\subsection{Особенности электронного и геометрического строения нитрозильных комплексов. Описание связи металл - нитрозил. Сравнение со связью M-CO. Природа связи M-CO и M-NO}

\subsubsection*{Электронное строение} похоже на CO, но с 1 e на $\pi^*$ орбитали. Донирование 3 е.

$\pi$-орбитали лежат низко

На $\pi^*$ есть1 e, который может участвовать в связывании

$NO^(+)$ изоэлектронен $CO$



\subsubsection*{Геометрия}
 может донировать 3 е или 1 е.

1е: 

угол $MNO$ - не 180 град., связь M-N длиннее

3е:

угол $MNO$ - ровно 180 град., связь M-N короче


\subsubsection*{Связь M-NO и M-CO}

\begin{tabular}{|l|l|l|}
	\hline
	электронов донируется & 1e или 3e & 2e\\
	\hline
	$\pi$-акцептор & редко & хорошо\\
	\hline
	ионность & отн. выскокая & малая\\
	\hline
\end{tabular}


Угол $MNO$ зависит от: природы лиганда; других лигандов металла; стерического фактора
