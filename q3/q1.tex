\subsection{Природа связи металл-лиганд. Основные положения теории кристаллического поля}

По методу валентных связей образование комплекса представляет собой реакцию между основанием Льюиса (лигандами $L$) и кислотой Льюиса (центральным атомом $М$) с формированием ковалентных связей $M$-$L$.

По ТКП к катиону приближается электроотрицательный атом, например $F^-$. Он будет садиться на металл с избыточной электронной плотностью. Анион доходит до оболочки инертного газа, электронная плотность выдавливается. Происходит поляризация. За счёт поляризации анион подойдёт ближе, и будет выигрыш в энергии.

Согласно ТКП, взаимодействие между переходным металлом и лигандами возникает вследствие притяжения между положительно заряженным катионом металла и отрицательным зарядом электронов на несвязывающих орбиталях лиганда. Теория рассматривает изменение энергии пяти вырожденных $d$-орбиталей в окружении точечных зарядов лигандов. По мере приближения лиганда к иону металла, электроны лиганда становятся ближе к некоторым $d$-орбиталям, чем к другим, вызывая потерю вырожденности. Электроны $d$-орбиталей и лигандов отталкиваются друг от друга как заряды с одинаковым знаком. Таким образом, энергия тех $d$-электронов, которые ближе к лигандам, становится выше, чем тех, которые дальше, что приводит к расщеплению уровней энергии $d$-орбиталей.

ТКП: Металл мы рассматриваем структурированно, а лиганд точечно и с отрицательным зарядом.

На расщепление влияют следующие факторы:
\begin{itemize}
\item Природа иона металла.

\item Степень окисления металла. Чем выше степень окисления, тем выше энергия расщепления.

\item Расположение лигандов вокруг иона металла.

\item Природа лигандов, окружающих ион металла. Чем сильнее эффект от лигандов, тем больше разность между высоким и низким уровнем энергии.
\end{itemize}