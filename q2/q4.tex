
\subsection{2.4. Обсудите все известные вам тенденции в изменении ионных радиусов, приведите примеры таких изменений} 

\par\bigskip


Поскольку в модели жёстких сфер ионы практически несжимаемы,
считается, что у ионов есть определённые радиусы, которые можно
экспериментально измерить. Во многих справочниках ионные
радиусы представлены с точностью до сотых ангстрема, так как в
подавляющем большинстве случаев ионные радиусы ионов не
изменяются при переходе от одной структуры к другой. Мы можем
определить расстояния между ионами, но чтобы понять, какую
часть составляет в этом расстоянии ионный радиус одного из
ионов, надо решать системы линейных уравнений, где данный ион
участвует в одном соединении, в другом, в $n$-ом…

\par\smallskip
\begin{center}
\textbf{Тенденции:}
\end{center}

1) \textbf{Фаза}

\par\smallskip

Сумма ионных радиусов двух ионов в газовой фазе больше, чем
в твёрдой фазе (в кристаллической фазе каждый ион
взаимодействует со множеством других ионов и сильнее
отдаляется от других).

\par\smallskip

Например, $LiF$ - существует и в газовой фазе, и в кристалле.
Расстояние $Li^+ F^-$ в газовой фазе равно $1.52 \textup{\AA}$, а в твёрдой $1.96 \textup{\AA}$. То есть, $\frac{d(gas)}{d(solid)}$ примерно равно $78\%$.

\par\smallskip

2) \textbf{Координационное число}

\par\smallskip

И для катионов, и для анионов ионные радиусы увеличиваются
с возрастанием координационного числа (КЧ). При повышении
КЧ сила отталкивания между противоионами возрастает и
заставляет их раздвигаться.

\par\smallskip

Например:

\par\bigskip	

$\;\;\;\;\;\;\;\;\;\;\;\;\;$ КЧ$=4$ $\;\;\;\;\;$ $R=53$ пм

$Al^{3+}$

$\;\;\;\;\;\;\;\;\;\;\;\;\;$  КЧ$=6$ $\;\;\;\;\;$ $R=67.5$ пм

\par\bigskip

3) \textbf{Изменение заряда}

\par\smallskip

При одинаковом КЧ ионный радиус катионов уменьшается при
повышении заряда, так как, во-первых, сам ион становится
меньше при потере электронной плотности, а во-вторых, с
повышением заряда катион в большей степени притягивает
анионы.

\par\smallskip

Для анионов, видимо, неоднозначно: с одной стороны,
увеличение заряда (по модулю) заставляет катионы сильнее
притягиваться, что должно уменьшать ионный радиус. С другой
стороны, увеличивается электронная оболочка аниона, как и
ионный радиус.

Например:

\par\bigskip	

$Mn^{4+}$ $\;\;\;\;\;$ КЧ$=4$ $\;\;\;\;\;$ $R=53$ пм	

$\;\;\;\;\;$

$Mn^{6+}$ $\;\;\;\;\;$ КЧ$=4$ $\;\;\;\;\;$ $R=39.5$ пм	

\par\bigskip
	
\textbf{Особенность:}

\par\smallskip

Некоторые пары элементов, расположенные в ПСХЭ по диагонали
друг относительно друга, имеют близкие ионные радиусы:

\par\smallskip

\begin{center}
\begin{tabular}[c]{llllllll}
	& 1 & 2 &  &  & $Li^+$ &  & $0.88 \textup{\AA}$ \\
	2 & \textcolor{red}{$Li$} & $Be$ &  &  &   &  &   \\
	3 & $Na$ & \textcolor{red}{$Mg$} &  &  & $Mg^{2+}$ &  & $0.86 \textup{\AA}$ 
\end{tabular}
\end{center}

\par\smallskip	

4) \textbf{Природа иона:} по группе

\par\smallskip

\begin{center}
\begin{tabular}[c]{lllll|llll}
	Ион & $F^-$ & $Cl^-$ & $Br^-$ &  &  & $Li^+$ & $Na^+$ & $K^+$ \\
	$R(\textup{\AA})$   & $1.36$ & $1.81$ & $1.95$ &  &  & $0.6$ & $0.95$ & $1.33$
\end{tabular}
\end{center}

\par\smallskip
	
Вниз по группе при одинаковом заряде и КЧ ионные радиусы
ионов $s$- и $p$-элементов увеличиваются, так как электронная
оболочка становится больше.

\par\smallskip

5) \textbf{Природа иона:} по периоду

\par\smallskip

У $s$- и $p$-элементов одного периода заряд иона возрастает слева
направо, что приводит к сильному уменьшению их ионных
радиусов:

\par\smallskip

\begin{center}
\begin{tabular}[c]{llllllll}
	&  &  & $Na^+$ &  & $Mg^{2+}$ &  & $Al^{3+}$ \\
	КЧ$=6$ &  &  &   &  &   &  &   \\
	&  &  & $116$ пм &  & $86$ пм &  & $67.5$ пм
\end{tabular}
\end{center}

\par\smallskip

В ряду лантаноидов наблюдается «лантаноидное сжатие»: при
увеличении порядкового номера размер ионов одинакового заряда
уменьшается из-за неполного экранирования заряда ядер
электронами $d$- и особенно $f$-подуровней:

\par\smallskip

\begin{center}
	\begin{tabular}[c]{lllll}
		$La^{3+}$ & --- & $Eu^{3+}$ & --- & $Lu^{3+}$ \\
		$1.20\textup{\AA}$ &     & $1.09\textup{\AA}$ &     & $0.99\textup{\AA}$
	\end{tabular}
\end{center}

\par\smallskip

6) \textbf{Знак заряда иона}

\par\smallskip

Анион обычно больше катиона (больше электронная оболочка).

\par\smallskip

\begin{center}
	\begin{tabular}[c]{llll}
		$Se^{2-}$ & $1.84 \textup{\AA}$ &  &   \\
		&   &  & КЧ$=6$ \\
		$Se^{4+}$ & $0.64 \textup{\AA}$&  &  
	\end{tabular}
\end{center}

\par\bigskip
\par\bigskip