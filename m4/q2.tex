
\subsection{Сильное и слабое кристаллическое поле. Магнитные и спектральные свойства комплексных соединений переходных металлов}

\subsubsection*{Сила поля лигандов}


Лиганды, вызывающие большое расщепление d-уровней, например $CN^-$ и $CO$, называются лигандами сильного поля. В комплексах с такими лигандами электронам невыгодно занимать орбитали с высокой энергией. Следовательно, орбитали с низкой энергией полностью заполняются до того, как начинается заполнение орбиталей с высокой энергией. Такие комплексы называются низкоспиновыми. Например, $NO_2^-$ — лиганд сильного поля, создающий большое расщепление. Все 5 d-электронов октаэдрического иона $[Fe(NO_2)_6]^{3-}$ будут располагаться на нижнем уровне $t_{2g}$.

Напротив, лиганды, вызывающие малое расщепление, например $I^-$ и $Br^-$, называются лигандами слабого поля. В этом случае легче поместить электроны в орбитали с высокой энергией, чем расположить два электрона в одной орбитали с низкой энергией, потому что два электрона в одной орбитали отталкивают друг друга, и затраты энергии на размещение второго электрона в орбитали выше, чем $\Delta$. Таким образом, прежде чем появятся парные электроны, в каждую из пяти d-орбиталей должно быть помещёно по одному электрону в соответствии с правилом Хунда. Такие комплексы называются высокоспиновыми. Например, $Br^-$ — лиганд слабого поля, вызывающий малое расщепление. Все 5 d-орбиталей иона $FeBr_6^{3-}$, у которого тоже 5 d-электронов, будут заняты одним электроном.

Энергия расщепления для тетраэдрических комплексов $\Delta_{tetr}$ примерно равна $\frac 49 Delta_{oct}$(для одинаковых металла и лигандов). В результате этого разность энергетических уровней d-орбиталей обычно ниже энергии спаривания электронов, и тетраэдрические комплексы обычно высокоспиновые.

\subsubsection*{Магнитные и спектральные свойства}
Диаграммы распределения d-электронов позволяют предсказать магнитные свойства координационных соединений. Комплексы с непарными электронами являются парамагнитными и притягиваются магнитным полем, а без — диамагнитными и слабо отталкиваются. Со спектральными свойствами - еще проще: чем выше $\Delta$, тем выше частота волны и меньше ее длинна.