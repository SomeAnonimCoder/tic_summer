\subsection{1.22 Соединения переходных металлов 6 группы. Геометрия молекул в зависимости от природы лигандов, их электронное строение, способы получения и химическое поведение.}
\begin{itemize}
	\item $Cr$: $+2, +3, (+4), (+5), +6$ \quad \quad $3d^54s^1$
	\item $Mo$: $+2, +3, +4, +5, +6$ \quad $4d^1 5s^1$
	\item $W$: $(+2), (+3), (+4), +5, +6$ \quad $4f^{14} 5d^4 6s^2$
\end{itemize}
Свойства:
\begin{itemize}
	\item $Cr$:
	\begin{itemize}
		\item $+ H_2SO_4 (\text{к})/HNO_3(\text{к}) = Cr_2(SO_4)_3 / Cr(NO_3)_3 + SO_4/NO_2 + H_2O $
		\item $+ KNO_3, KOH \text{(расплав)} = K_2CrO_4 + KNO_3 + H_2O$
		\item $+ HCl = CrCl_2 + H_2$
		\item $+ KOH \not =$
		\item $+ H_2O \text{(пар)} = Cr_2O_3 + H_2$	
		\item $+ O_2 = Cr_2O_3$
		\item $+ Hal = CrHal_3 $
		\item $+ S, \,\, P, \,\, N_2, \,\, C, \,\, As, \,\, B = CrS, \,\, CrP_2, \,\, CrN,\,\, Cr_7C_3, \,\, CrAs,\,\, CrB $	
	\end{itemize}
	\item $Mo, W$:
	\begin{itemize}
		\item $+$ кислоты неок. $ \not = $
		\item $+ Hal = MoF_6, \,\, WCl_6 \,\, MoCl_5 \,\, WBr_5 $
		\item $+ O_2 = MoO_3$
		\item $+ HNO_3 + HF/HCl = H_2\left[WF_8 \right]/MoO_2Cl_2 + NO_2/NO + H_2O $		
	\end{itemize}
\end{itemize}
$Cr$: хромистый железняк $FeCr_2O_4$
\[
FeCr_2O_4 + 4C = Fe + 2Cr + 4CO \text{ (феррохром)}
\]
\begin{itemize}
	\item Вскрытие минерала
	\item $Cr_2O_3 + 2 Al = Al_2O_3 + 2 Cr$ (технический хром)
\end{itemize}
$Mo$: молибденит $MoS_2$
\begin{align*}
MoS_2 + O_2 &= MoO_3 + SO_2 \\
MoO_3 + H_2 &= Mo + H_2O
\end{align*}
$W$: вольфрамит $(Fe, Mn)WO_4$
\begin{itemize}
	\item Вскрытие минерала
	\item $WO_3 + 3 H_2 =W + 3 H_2O$
\end{itemize}
\textbf{Степень окисления $+2, +3$} - $Mo, W$ низкие с.о.\\
Для $Mo, W$ мало монноядерных комплексов, много с $M-M$, в основном кластеры. \\
$d^3$- окт.: $ \left[Mo^{+3}(H_2O)_6 \right]^{3+} \quad \left[Mo^{+3}(HCOO)_6 \right]^{3-} \quad \left[Mo^{+3}(HPO_4)_4 \right]^{2-}$ \\ 
$\left[Mo^{+3}(SCN)_6 \right]^{3-} \quad \left[Mo^{+3}Cl_6 \right]^{3-} \quad \left[W_2^{+3}Cl_9 \right]^{3-} $ \\
$ WCl_2 + 2Cl_2 = CCl_4 = W_6Cl_{16} $ \\
$ MoO_3 + 9CO = Mo(CO)_6 + 3CO_2 $
\textbf{Степень окисления $+2, +3$} - $Cr^{2+}$ - сильный восстановитель: \\
$CrCl_2 + 2 NaCp = Cr(Cp)_2 + 2 NaCl$ \\
$d^4$, эффект Яна-Теллера: $ \left[CrHal_3 \right]^{-} \quad \left[CrHal_4 \right]^{2-} \quad \left[Cr(H_2O)_6 \right]^{2+}  $
\textbf{Степень окисления $0$} - $Cr$ \\
\begin{align*}
CrCl_3 + Al + CO &= THF = Cr(CO)_6 + AlCl_3 \\
CrCl_3 + Al + 2C_6H_6 &= Cr(C_6H_6)_2 + AlCl_3
\end{align*}
\textbf{Степень окисления $+3$} - $Cr^{+3}$ - амфортерные свойства \\
$Cr_2O_3 + HNO_3 / KOH \not = $ - химически инертный \\
$Cr_2O_3 + KOH + KNO_3 = t = K_2CrO_4 + KNO_2 + H_2O$ \\
Много комплексов: инертны и устойчивы \\
\[
\left[Cr(H_2O)_6 \right]^{3+} \quad \left[CrCl(H_2O)_5 \right]^{2+} \quad \left[Cr(acac)_3 \right] \quad \left[Cr(ox)_3 \right]^{3-} \quad \left[Cr(CN)_6 \right]^{3-}	
\]
$d^3$, нет эффекта Яна-Теллера \\
Окислитель: $ CrCl_3 + Zn = CrCl_2 + ZnCl_2 $ \\
Восстановитель: $ Cr_2(SO_4)3 + MnO_2 + H_2O = H_2CrO_4 + MnSO_4 $ \\ 
\textbf{Степень окисления $+4, +5$} \\
$MoO_2, WO_2$ - искаженная структура рутила \\
$ \left[MoCl_6 \right]^{2-} \quad \left[Mo(CN)_8 \right]^{4-} \quad \left[W_3O_2(OAc)_6(H_2O)_3\right]^{2-} \text{(хелатный)} $ \\
$ \left[MoCl_3L_2 \right] $ где $L = THF, PEt_3,  bpy$ \\
$\left[MoOCl_5 \right]^{2-}$ устойчивы галогениды и оксогалогениды \\
$CrF_5$ - тригональная бипирамида \\
$CrF_4$ - тетраэдр \\
\textbf{Степень окисления $+6$} - $Cr^{+6}$ - окислитель (самая устойчивая с.о.)\\
$CrO_3 + 2KOH = K_2CrO_4 + H_2O$ \\ 
Хроматы сильные окислители в кислоте и слабые в щелочи. \\
$\left[CrO_4\right]^{2-}  $ - тетраэдр \\
$ Cr_2O_3 $ - цепочка из тетраэдров \\
$Mo^{+6}, W^{+6}$ - плохие окислители \\
Полимеризация \\
$\left[MoO_4\right]^{2-} \quad \left[WO_4\right]^{2-} $ - тетраэдры \\
$Cr$ и $Mo$ в степени окисления $+6$ и $W$ в степени окисления $+5$ образуют изо- и гетерополисоединения (анионы Кеггина, молибденовый синий)

