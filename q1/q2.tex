\subsection{Ионная связь. Строение ионных соединений в твердой фазе и в растворе.}

В основе ионных соединений лежит кулоновское взаимодействие. У
ионных кристаллов в узлах решётки находятся положительные и
отрицательные ионы, в отличие от атомов или молекул в узлах
ковалентных кристаллов. Ионы в узлах ионного кристалла
расположены таким образом, что силы притяжения между
разнозаряженными ионами максимальны, а силы отталкивания
между одинаково заряженными ионами минимальны. 

Согласно модели жёстких сфер, ионы рассматриваются как практически
несжимаемые жёсткие сферы, почти не смещающиеся со своих
позиций. Невозможно обнаружить сходство свойств иона и атома,
образовавшего этот ион. Свойства катиона не зависят от свойств
аниона и наоборот. В ионных соединениях невозможно выбрать
направление связи, нет валентных углов. Основные требования при
образовании ионных соединений заключаются в том, что атомы
металла должны иметь относительно низкий потенциал
ионизации, а атомы неметаллов или радикалы - сравнительно
высокое сродство к электрону. 

В состав ионного соединения должны входить элементы с сильно различающимися
значениями электроотрицательности. Общая электронная пара
локализуется у атома с большей электроотрицательностью. В
отличие от ковалентной связи, разделение зарядов практически
полное. 

Тем не менее, ионных на 100 процентов соединений не существует,
поскольку за «идеал» ионного взаимодействия берётся связь
протона и электрона, там разделение зарядов 100-е. 

Степень разделения зарядов можно определять по величине
дипольного момента молекулы, который измеряется в Дебаях (Д).
Помимо кулоновского притяжения двух находящихся рядом
противоположно заряженных ионов, будет осуществляться и
взаимодействие, обусловленное взаимной поляризацией. Чем
меньше по размеру и более высоко заряжен катион, тем больше он
будет стремиться нарушить распределение заряда в соседнем
анионе. Поляризуемость аниона растет с ростом его размера и
заряда (по модулю). В основном смотрят на деформацию именно
анионов катионами, а не наоборот, поскольку анионы по размеру
больше катионов. Если при ионной связи поляризация анионов
настолько велика, что дает заметное увеличение электронной
плотности между ядрами, то уже можно рассматривать это как
случай ковалентной связи.

\subsubsection*{\textbf{Строение и свойства ионных соединений в твердой фазе}}

Представляют собой бесконечную периодическую решётку. В
идеале, кристалл состоит из бесконечного количества
элементарных ячеек. В твёрдой фазе очень много ионных
взаимодействий. Электрическая проводимость чаще всего низкая,
потому что ионы в узлах решётки закреплены и не могут свободно
перемещаться. Ионные соединения имеют высокие температуры
плавления, потому что ионные связи обычно сильны и
ненаправленны (распространяются во всех направлениях). Ионные
соединения обычно твёрдые, но хрупкие. Твёрдость объясняется
невозможностью образования кратных связей вследствие
разделения ионов в пространстве и отсутствия теплового движения
ионов. Хрупкость объясняется природой ионной связи: даже при
относительно небольшом сдвиге ионов возникают контакты анионанион и катион-катион и вместо сил притяжения появляются силы
отталкивания, вследствие чего кристалл раскалывается.

\subsubsection*{\textbf{Пример проявления поляризации}}

$$CaF_2 - CaCl_2 - CaBr_2 - CaI_2$$

В этом ряду слева направо уменьшается температура
плавления соединения, потому что увеличиваются
радиусы анионов, что увеличивает их
поляризуемость. Это снижает энергию
кристаллической решётки, так что для расплавления
вещества требуется меньше энергии.

\subsubsection*{\textbf{Строение и свойства ионных соединений в растворе}}

Ионные соединения в растворе взаимодействуют с полярными
молекулами растворителя. За счёт большой энергии сольватации
происходит выигрыш в энергии. В растворе можно обнаружить
следующие ассоциаты:


1) Сольватированные ионы( реально встречаются только при низких концентрациях)

$$M^+(solv) - - - - -X^-(solv)$$

2) Сольватно-разделенные ионные пары(сольватированные ионы разделены молекулой растворителя)

$$M^+(solv) || X^-(solv); ||=(solv)$$

3) Контактные ионные пары

$$(solv)M^+X^-(solv)$$

В растворе ионные соединения имеют высокую проводимость,
поскольку в этих веществах имеются ионы, которые могут
свободно двигаться под действием электрического поля. Ионные
соединения заметно растворимы в полярных растворителях с
высокой диэлектрической проницаемостью. Энергия
взаимодействия двух заряженных частиц в некоторой среде обратно
ей пропорциональна:


$$E = \frac{q^+q^-}{4\pi\epsilon r}$$
