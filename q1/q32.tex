\subsection{ Карбонилгалогениды и карбонилат-анионы. Способы получения и условия стабильности, геометрия, электронное строение, свойства}

\subsubsection*{Полуение}

$$Fe(CO)_5 + J_2 \Rightarrow Fe(CO)_4J_2 + CO$$

$$MN_2(CO)_{10} + J_2 \Rightarrow 2 Mn(CO)_5J \Rightarrow Mn_2(CO)_8J_2(bridge iodides)$$

$$Mn_2(CO)_8J_2 + Py \Rightarrow Mn_2(CO)_8JPy + J-$$

$$2PtCl_2 + 2CO \Rightarrow [Pt(CO)Cl_2]_2$$

\subsubsection*{Прекурсоры для получения соединений со связью $M-M$}

$$Mn(CO)_5Br + [Re(CO)_5]^- \Rightarrow (CO)_5Mn-Re(CO)_5 + Br^-$$

\subsubsection*{Строение}

Поздние переходные металлы не образуют устойчивых бинарных карбонилов(не совсем корректный термин!), вместо этого образуются карбонилгалогениды.


\subsubsection*{Условия устойчивости}

Как и везде, 18 e + стерика
