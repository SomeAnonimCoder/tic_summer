
\subsection{Олефинкарбонильные комплексы переходных металлов. Способы получения и условия стабильности, геометрия, электронное строение, свойства. Сравнение параметров связи металл-олефин для различных металлов}

\subsubsection*{Получение}

$$Fe(CO)_5 + H_2C=CH-CH-CH_2 = Fe(CO)_3(H_2C=CH-CH-CH_2) + 2CO$$
$$Cr(CO)_4 + CH_2=CH_2 = Cr(CO)_4(C2H4)$$

\subsubsection*{Условие стабильности}

18 e и стерический фактор

\subsubsection*{Электронное строение и геометрия}

Связь металл-олефин описывается моделью ДЧД, то есть совокупностью прямого и обратного донирования. Если обратного донирования очень много, то порядок $\pi$-связи в олефине близок к 0, то есть атомы углерода имеют $sp^3$-гибридизацию.

Для подробной информации см. вопрос 3.11


